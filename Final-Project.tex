\documentclass{article}

\usepackage{fancyhdr}
\usepackage{extramarks}
\usepackage{amsmath}
\usepackage{amsthm}
\usepackage{amsfonts}
\usepackage{tikz}
\usepackage[plain]{algorithm}
\usepackage{algpseudocode}

\usetikzlibrary{automata,positioning}

%
% Basic Document Settings
%

\topmargin=-0.45in
\evensidemargin=0in
\oddsidemargin=0in
\textwidth=6.5in
\textheight=9.0in
\headsep=0.25in

\linespread{1.1}

\pagestyle{fancy}
\lhead{\hmwkAuthorName}
%\chead{\hmwkClass\ (\hmwkClassInstructor\ \hmwkClassTime): \hmwkTitle}
\chead{\hmwkClass: \hmwkTitle}
\rhead{\hmwkClassInstructor}
\lfoot{\lastxmark}
\cfoot{\thepage}

\renewcommand\headrulewidth{0.4pt}
\renewcommand\footrulewidth{0.4pt}

\setlength\parindent{0pt}

%
% Create Problem Sections
%

\newcommand{\enterProblemHeader}[1]{
    \nobreak\extramarks{}{Problem \arabic{#1} continued on next page\ldots}\nobreak{}
    \nobreak\extramarks{Problem \arabic{#1} (continued)}{Problem \arabic{#1} continued on next page\ldots}\nobreak{}
}

\newcommand{\exitProblemHeader}[1]{
    \nobreak\extramarks{Problem \arabic{#1} (continued)}{Problem \arabic{#1} continued on next page\ldots}\nobreak{}
    \stepcounter{#1}
    \nobreak\extramarks{Problem \arabic{#1}}{}\nobreak{}
}

\setcounter{secnumdepth}{0}
\newcounter{partCounter}
\newcounter{homeworkProblemCounter}
\setcounter{homeworkProblemCounter}{1}
\nobreak\extramarks{Problem \arabic{homeworkProblemCounter}}{}\nobreak{}

%
% Homework Problem Environment
%
% This environment takes an optional argument. When given, it will adjust the
% problem counter. This is useful for when the problems given for your
% assignment aren't sequential. See the last 3 problems of this template for an
% example.
%
\newenvironment{homeworkProblem}[1][-1]{
    \ifnum#1>0
        \setcounter{homeworkProblemCounter}{#1}
    \fi
    \section{Problem \arabic{homeworkProblemCounter}}
    \setcounter{partCounter}{1}
    \enterProblemHeader{homeworkProblemCounter}
}{
    \exitProblemHeader{homeworkProblemCounter}
}

%
% Homework Details
%   - Title
%   - Due date
%   - Class
%   - Section/Time
%   - Instructor
%   - Author
%

\newcommand{\hmwkTitle}{Final Project}
\newcommand{\hmwkDueDate}{December 11, 2018}
\newcommand{\hmwkClass}{Differential Equations}
\newcommand{\hmwkClassTime}{Section 3000}
\newcommand{\hmwkClassInstructor}{Dr. Darrel Thoman}
\newcommand{\hmwkAuthorName}{\textbf{Brent Pierce} \and \textbf{Hayden White}}

%
% Title Page
%

\title{
    \vspace{2in}
    \textmd{\textbf{\hmwkClass:\ \hmwkTitle}}\\
    %\normalsize\vspace{0.1in}\small{Due\ on\ \hmwkDueDate\ at 3:10pm}\\
   \vspace{0.1in}\large{\textit{\hmwkClassInstructor\ \hmwkClassTime}}
    \vspace{3in}
}

\author{\hmwkAuthorName}
\date{}

\renewcommand{\part}[1]{\textbf{\large Part \Alph{partCounter}}\stepcounter{partCounter}\\}

%
% Various Helper Commands
%

% Useful for algorithms
\newcommand{\alg}[1]{\textsc{\bfseries \footnotesize #1}}

% For derivatives
\newcommand{\deriv}[1]{\frac{\mathrm{d}}{\mathrm{d}x} (#1)}

% For partial derivatives
\newcommand{\pderiv}[2]{\frac{\partial}{\partial #1} (#2)}

% Integral dx
\newcommand{\dx}{\mathrm{d}x}

% Approximate symbol
\newcommand*{\approxident}{%
  \mathrel{\vcenter{\offinterlineskip
  \hbox{$\sim$}\vskip-.35ex\hbox{$\sim$}\vskip-.35ex\hbox{$\sim$}}}}

% Alias for the Solution section header
\newcommand{\solution}{\textbf{\large Solution}}

% Probability commands: Expectation, Variance, Covariance, Bias
\newcommand{\E}{\mathrm{E}}
\newcommand{\Var}{\mathrm{Var}}
\newcommand{\Cov}{\mathrm{Cov}}
\newcommand{\Bias}{\mathrm{Bias}}

\begin{document}

\maketitle

\pagebreak

\begin{homeworkProblem}[9]
	A motorboat weighs 32,000 lb and its motor provides a thrust of 5000 lb.
	Assue that the water resistance is 100 pounds for each foot per second of the speed \(v\) of the boat. Then

	\[
		\begin{split}
			1000 \frac{dv}{dt} &= 5000 - 100{v}
		\end{split}
	\]

	If the boat starts from rest, what is the maximum velocity that it can attain?
	\\

	\textbf{Solution}
	\\

	We begin by simplifying the formula used in the problem
	
	\[
		\begin{split}
			1000 \frac{dv}{dt} &= 5000 - 100{v}
			\\
			10 \frac{dv}{dt} &= 50 - {v}
		\end{split}
	\]

	We then integrate to find velocity as a function off time

	\[
		\begin{split}
			\int \frac{dv}{(50-{v})} = \frac{1}{10} \int dt
			\\
			- \ln (50-{v}) = \frac{1}{10} t + {C}
			\\
			\ln (50-{v}) = -\frac{1}{10} t - {C}
			\\
			50 - {v} = C_1 e^{-\frac{t}{10}}
			\\
			v(t) = 50 + C_2 e^{-\frac{t}{10}}
		\end{split}
	\]

	Given that initial velocity is 0, we can then use our function to find \(C_2\)

	\[
		\begin{split}
			v(0) = 50 + C_2 e^{-\frac{0}{10}}
			\\
			0 = 50 + C_2 e^{0}
			C_2 = -50
		\end{split}
	\]

	Substituting back into our equation, we obtain

	\[
		\begin{split}
			v(t) = 50 - 50 e^{-\frac{t}{10}}
		\end{split}
	\]

	Finally, we examine the limit as \(t\) approaches infinity to find the maximum velocity

	\[
		\begin{split}
			\lim_{t \to \infty} 50-50e^{-\frac{t}{10}} = 50 - 50(0) = 50
		\end{split}
	\]

	Therefore,

	\[
		\begin{split}
			{v_{max}} \approx 50 \frac{ft}{sec}
		\end{split}
	\]

\end{homeworkProblem}

\pagebreak

\end{document}
